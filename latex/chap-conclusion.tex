\chapter{Conclusion}\label{chap:conc}

Recalling what was said in \autoref{chap:intro}, the objectives of this thesis
were to explore how eBPF could be used for monitoring experiments in wireless
networks, and to create a program that would showcase this.

This work was very exploratory, as it was necessary to examine the code of the
Linux kernel which is not only very extensive but also very complex, study
\ac{IEEE} 802.11s mesh networks, as well as learn eBPF, all of which are fields
where we had little to no prior experience. Most of the documentation of eBPF
and the tools and frameworks around it assume some familiarity with the Linux
source code, how it works, and its most used structures, which was not the case
for us.

As for the programs developed, we accomplished our objective, which was to
create a tool that could demonstrate how eBPF could be used in experiments of
wireless networks. Our service program can detect changes in the mesh path table
of several systems, and its output files can be then used in conjunction with
outputs from other systems to be analysed with the central program to get a
timeline of the activity in an entire \ac{IEEE} 802.11s mesh network.

It is worth mentioning that in our program we use probes as a necessity, which
are not as stable as tracepoints. This means that an update to the kernel that
changes the functions we probe could result in our program not working
correctly, and an update to fix these probes would be required.


\section{Future Work}

Although we were able to reach our objectives, we believe our programs could be
improved with more features and fixes to compromises that had to be made due to
the lack of time available. The following paragraphs contain some ideas and
possible solutions for the implementation.

The first one is the presence of the probe in the function
\texttt{mesh\_plink\_deactivate}. This probe ignores some actions that can
modify the mesh path table, so removing it would be an improvement. To remove
it, a deeper analysis of the functions that call
\texttt{mesh\_plink\_deactivate} would be needed, to take into account all the
possible call stacks.

One of the biggest issues we wanted to fix but could not for lack of time was
the way we sort all the events in the \textbf{central} program. We use the
timestamp captured along with the path information for sorting events, but since
the clocks of the several computers in a network being monitored will not be
perfectly synchronized, it is not a reliable metric. One way we think this could
be solved would be to use the ordering of packets instead, using two packet
capture files where one has a packet that causes a change to a mesh path table
being sent, and another has that same packet being received, and using that
packet as a basis for sorting events between the two stations where these
packets were captured. The timestamp of the events could additionally be used to
shift the events in a station that were not caused by a packet in relation to
the events in other stations, using the packet previously mentioned as a
reference point in time.

With hindsight, we can see that the decision to switch to a single probe at
packet reception when we switched from \ac{BCC} \ac{CO-RE} was not ideal. We did
it mostly because the code became shorter and much easier to comprehend. Still,
although we never detected it, there is a chance that some function could change
the content of a packet. If we had time we would revert to two probes, with the
entry probe inserting the important content of the packet to the BPF map, have
the probes at each action retrieve the information from the BPF map and
submitting it to user-space, and have the exit probe delete anything in the BPF
map for the cases where the thread did not alter the mesh path table.

Something that we think would be fun, but also help visualize bigger networks,
is to create a graph showing the mesh paths in each event. This would basically
be an easy-to-interpret timeline of the network. Being able to click on each
station to see the events that it captured, as well as each mesh path to see its
information, along with which packet (if applicable) caused it would be great as
well. Because the events already store the source, nexthop, and destination of
each path, this would only require updating the central program.

Another thing that would possibly help with bigger networks is a file that
contains the information of a whole network. Right now, the \textbf{central}
program takes the files of all the stations and sorts their events every time it
is executed. We never experienced a slow sorting process, with the program
always being opened almost immediately, but considering that our tests were
always quite short and involved at most three stations, it is possible that
bigger networks and longer tests would cause a slow-down. We think the
\textbf{central} program could have a flag that generates a file that contains
all the events of all stations already sorted, to be used in other runs, instead
of the original files.

Something that would be nice to have would be a way for the central program to
be able to retrieve the information gathered by the different nodes being
monitored automatically, instead of relying on manual file transfers from those
nodes.

The way we sort the events in the \textbf{central} program, although simple, is
not very efficient given that it increases the lists of events for each station,
filling them mostly with empty elements, which still occupy as much space in
memory as the structures for events. A better approach would be to use would be
preferable.

Something that could be improved is the fields of paths that the tool captures.
Currently, it captures only the destination and nexthop of paths, but other
fields can be captured, such as the metric value and the sequence number, as was
noted in the last paragraph of \autoref{sect:valid}.

One last thing that we think would help with analysis would be to show the time
between events. This would enable users to easily determine how much time has
passed between events, and could be implemented using the timestamps available
in the events.
