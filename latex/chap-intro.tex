\chapter{Introduction}\label{chap:intro}

Monitoring and observability have always been important aspects in the field of
computer systems. They can be found everywhere, being used to determine the
performance and health of systems by administrators, used by programmers in the
form of debuggers, and even in research, to analyse new technologies and verify
that they are working as expected.

Several monitoring tools have been created over the years for all kinds of
purposes and hardware, but as components and software become more advanced,
these tools end up having to be replaced or updated. One of these newer tools
that has been gaining popularity in the last few years is eBPF.

eBPF is a technology that started out as the \ac{BPF}, which was a tool for
writing efficient packet filters. It has since been improved, having received
new capabilities, and also updated in order to take advantage of newer hardware,
transforming into eBPF. eBPF allows users to write programs that run on certain
events of the Linux kernel, allowing not only for monitoring of its internals,
but also enhancing its capabilities without the need to use kernel modules, all
of this while still ensuring the security and stability of the kernel. Because
eBPF can not only be used to access the internals of a system, but also
intercept the packets that system sends and receives, eBPF can be used to
monitor a whole network of systems.

One example of how eBPF can be used, is to monitor a network of systems in
\ac{IEEE} 802.11s wireless mesh networks, by probing the functions in the Linux
kernel related to these mesh networks in each of the systems in said networks.


\section{Aims and Objectives}

The main objective of this work will be to study eBPF, seeing what it can and
can not do, and determining how it can be used for network observability in
experiments, particularly in \ac{IEEE} 802.11s networks. We will then proceed
with the development of a tool that uses eBPF to demonstrate the type of
monitoring applications that can be built with this technology, revealing the
exploratory work that was done in order to realise this tool.


\section{Organization}

This thesis will start with the background in \autoref{chap:stat}, where we will
go over \ac{IEEE} 802.11 and \ac{IEEE} 802.11s mesh networks, as well as what
eBPF is in more detail, and explaining the different tools that can used to take
advantage of this technology.

Next, \autoref{chap:expl} will cover the exploratory part of this work, where we
mention the steps we took to understand how the Linux kernel's implementation of
the data link layer for \ac{IEEE} 802.11 networks (mac80211) deals with
\ac{IEEE} 802.11s mesh networks, mentioning some of the tests used for this
purpose and to get some experience with writing eBPF programs.

In \autoref{chap:devel} we talk about the development of the program that was
written to serve as an example of a tool that uses eBPF for network monitoring,
going over each section of its code. We also introduce the companion program
that was created to view the results generated by the main program.

Some of the tests that were used during development are explained in detail in
\autoref{chap:tests}, where we analyse what was learned from them, and the
decisions taken based on their results, including the final test that was
performed with the finished program running in real hardware.

Finally, in \autoref{chap:conc} we take a look at what we managed to accomplish,
and the limitations we had to settle for. We also mention parts of our programs
that could be improved upon, together with possible solutions.

There is no ``State of the Art'' chapter as there was no state of the art
surrounding the use of eBPF for monitoring \ac{IEEE} 802.11s mesh networks at
the start of this work.
