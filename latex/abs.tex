\prefacesection{Sworn Statement}

I, Carlos Tiago Gomes Pinto, born in 00/00/0000, resident in PLACE, phone number
000000000, of Portuguese nationality, bearer of Identification Card 00000000,
enrolled in the Master Degree Network and Information Systems Engineering at the
Faculty of Sciences of the University of Porto hereby declare, in accordance
with the provisions of paragraph a) of Article 14 of the Code of Ethical Conduct
of the University of Porto, that the content of this dissertation reflects
perspectives, research work and my own interpretations at the time of its
submission.

By submitting this dissertation, I also declare that it contains the results of
my own research work and contributions that have not been previously submitted
to this or any other institution.

I further declare that all references to other authors fully comply with the
rules of attribution and are referenced in the text by citation and identified
in the bibliographic references section. This dissertation does not include any
content whose reproduction is protected by copyright laws.

I am aware that the practice of plagiarism and self-plagiarism constitute a form
of academic offence.

Carlos Tiago Gomes Pinto

30/09/2022


\prefacesection{Abstract}

Being able to monitor the changes in a network is an important part of
experimentation and troubleshooting in the investigation field. In this work we
explore eBPF, a technology available in the Linux kernel that can be used not
only for monitoring, but also for the creation of monitoring tools, as well as
the different libraries and frameworks available to help with eBPF usage.

Using eBPF, we develop a tool for monitoring the creation, modification, and
removal of paths in 802.11s mesh networks in Linux, for use with wireless
interfaces that use the softMAC implementation included in the kernel, mac80211.
The tool captures events that change paths and associate them with the packets
that triggered the changes, whenever applicable. It provides an interactive
graphical user interface that gathers information collected in the different
nodes and presents it to the user.
Along with the development, we also present the tests and work that was done in
the exploration of eBPF and the mac80211 subsystem of the Linux kernel.


\prefacesection{Resumo}

A capacidade de monitorizar mudanças numa rede de computadores é uma parte
importante da experimentação e resolução de problemas na área da investigação.
Neste trabalho exploramos o eBPF, uma tecnologia disponível no kernel Linux que
pode não só ser usada para monitorização, mas também para a criação de
ferramentas de monitorização, e as diferentes bibliotecas e \textit{frameworks}
disponíveis para ajudar com o uso do eBPF.

Usando eBPF, criamos uma ferramenta para monitorizar a criação, modificação, e
remoção de caminhos em redes \textit{mesh} 802.11s no Linux, para uso com
interfaces sem fios que usam a implementação softMAC incluída no kernel,
mac80211. A ferramenta captura eventos que alteram caminhos e associa-os com os
pacotes que provocaram as mudanças, sempre que aplicável. Fornece também uma
interface gráfica que junta a informação gerada nos diferentes nós e apresenta-a
ao utilizador. Em conjunto com o desenvolvimento, também apresentamos os testes
e trabalho realizados na exploração do eBPF e do subsistema mac80211 do kernel
Linux.


\prefacesection{Acknowledgements}

I want to thank first and foremost my advisors Rui Prior and Eduardo Soares for
the tremendous help they have given me throughout these last twelve months. I
would also like to thank the Instituto de Telecomunicações (UIDB/50008/2020) for
hosting this work.

I also want to thank my parents for the support they have given me throughout my
whole life, and my friends for the joyous times we spent together.
